\documentclass[11pt]{article}% 

% Import the packages and the commands

% main.mydef.tex

% ============================
% Packages
%
% Feel free to add your own packages
% ============================
\usepackage{pythontex} % Used to compile Python code within LaTeX
\usepackage{nopageno} % Used to remove the number of page in the document
\usepackage{graphicx} % Used to include images in the document
\usepackage{booktabs} % Used to make tables, specially the ones exported from pandas
% ============================
% Commands
%
% Feel free to add your own commands
% and to modify the existing ones
% ============================

% Command \includepythongraphics with options
\newcommand{\includepythongraphics}[2][]{%
    \IfFileExists{#2}{%
        \includegraphics[#1]{#2}%
    }{%
        % Error message if image not found
        \textbf{Image not found:} #2%
    }%
}

\begin{document}

\begin{pycode}
import math
import numpy as np
from matplotlib.pylab import plt    
\end{pycode}

% If you execute an external script. Make sure that the path to any files are relative to the main.tex file.
\pyc{exec(open('sources/script.py').read())}

% Generation of LaTeX Code based on a DataFrame
\begin{pycode}
df_head = df.head(10)
latex_table = df_head.to_latex()
\end{pycode}
    
% Show the generated table
The generated table is :
\begin{center}
\pyc{print(latex_table)}
\end{center}


% This block allows you to use python instructions but not printing the code in the document.
% Make sure to not indent the python code like you would do in LaTeX.
% Keep the indent of the python code as it is in the python file.
\begin{pycode}
x = math.sqrt(987.27)
# print("Square root :",  x)
\end{pycode}

\vspace{5em}

% The \py{} command allows you to write a python instruction without creating a LaTeX block.
The variable is $x=\py{x}$


\begin{pycode}
x = np.linspace(5,6,100)
# List of points :
xx = [5.13 , 5.7, 5.48, 5.7, 5.66, 5.84, 5.5, 5.69, 5.44, 5.84, 5.82, 5.88, 5.61, 5.65, 5.21, 5.37, 5.46, 5.33, 5.15, 5.74]
yy = [19.47, 21.67, 20.98, 21.46 , 21.6, 22.29, 20.84, 21.46, 20.8, 21.56, 22.02 , 22.54, 21.2, 21.39, 20.49, 20.46, 20.72, 20.07 , 20.24, 21.6]

# Make sure to be good with matplotlib and numpy to have a good result
plt.figure(figsize=(4,4), dpi=100)
a,b = np.polyfit(xx , yy , 1)
plt.plot(xx , yy , "o")
plt.plot(x , a*x+b)
plt.savefig('img/graphique.png',bbox_inches='tight')
\end{pycode}


\includepythongraphics{img/graphique.png}


\vspace{5em}

% This block allows you to use python instructions and printing the code in the document.
% Notes :
% - The variables, functions and libs aren't share between pycode and pyconsole.
% - The python code is executed in a console, like you'll do in the terminal.
\begin{pyconsole}
# Let's see the behavior of the python console
# You'll see that x is not defined
print(x)
result = 0
print("result =", result)
for i in range(10):
    result += i

print("result =", result)
# Make sure to add a blank line at the 
# end of the python code, specially if you use a for loop.
\end{pyconsole}


\end{document}